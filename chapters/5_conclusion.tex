\newpage
\section{CONCLUSION AND RECOMMENDATIONS}

\subsection{Conclusion}
The internship at NTQ Solutions has been a transformative experience, bridging the gap between academic theory and industrial application. The "Baby Cry Monitor v2.2" project evolved from a standalone prototype into a comprehensive IoT ecosystem, integrating Edge AI, Cloud Computing, and Mobile technology.

Key achievements include:
\begin{itemize}
    \item \textbf{Full-Stack IoT Implementation:} Successfully architected and deployed a complete pipeline connecting Raspberry Pi sensors, a FastAPI cloud backend, and a React Native mobile interface.
    \item \textbf{Performance Optimization:} Mastered system profiling techniques to implement the "Zero Disk I/O" strategy, reducing latency by over 90\%.
    \item \textbf{Advanced Data Management:} Leveraged TimescaleDB to handle high-frequency sensor data efficiently, enabling real-time analytics.
    \item \textbf{Cross-Platform Development:} Delivered a polished mobile user experience with robust real-time alert capabilities.
\end{itemize}

However, certain limitations remain:
\begin{itemize}
    \item \textbf{Hardware Cost:} The reliance on Raspberry Pi 4 increases the Bill of Materials (BoM).
    \item \textbf{Connectivity Dependence:} The system currently requires an active internet connection for the mobile app to receive alerts; a local LAN fallback mode is not yet implemented.
\end{itemize}

\FloatBarrier

\subsection{Recommendations}
\subsubsection{For NTQ Solutions}
\begin{itemize}
    \item \textbf{Edge AI Migration:} Investigate porting the YOLO model to lower-cost microcontrollers like the ESP32-S3 or K210 to reduce production costs.
    \item \textbf{Federated Learning:} Explore federated learning techniques to update AI models locally on user devices, enhancing privacy by reducing the need to upload audio data to the cloud.
\end{itemize}

\subsubsection{For Vietnam Japan University}
\begin{itemize}
    \item \textbf{Curriculum Enhancement:} Incorporate modules on Cloud-Native IoT architectures and Cross-Platform Mobile Development to better prepare students for modern tech stacks.
    \item \textbf{Industry Collaboration:} Continue fostering relationships with industry leaders to provide students with exposure to real-world R\&D challenges.
\end{itemize}

\FloatBarrier

\newpage
\section{SELF-REFLECTION}

\subsection{Lessons Learned}
This internship provided invaluable insights into the professional software development lifecycle:
\begin{enumerate}
    \item \textbf{System Thinking:} I learned to view problems holistically, understanding how a bottleneck in the Edge device affects the User Experience in the mobile app.
    \item \textbf{Technical Versatility:} Working across Python (Backend/AI), C++ (Optimization), and TypeScript (Mobile) enhanced my adaptability and ability to learn new technologies rapidly.
    \item \textbf{Professional Standards:} Participating in Agile sprints and code reviews instilled the importance of writing clean, maintainable, and well-documented code.
\end{enumerate}

\subsection{Future Aspirations}
Building on this foundation, my future goals are:
\begin{itemize}
    \item To pursue a career as a Full-Stack IoT Engineer, specializing in the intersection of Embedded Systems and Cloud AI.
    \item To conduct further research into Edge Computing architectures that prioritize user privacy in smart home applications.
    \item To apply the practical knowledge gained from this internship to my graduation thesis, ensuring it addresses real-world problems with viable solutions.
\end{itemize}

\FloatBarrier

\newpage
\section*{REFERENCES}
\addcontentsline{toc}{section}{References}

\begin{enumerate}[label={[\arabic*]}]
    \item Redmon, J., Divvala, S., Girshick, R., \& Farhadi, A. (2016). \textit{You Only Look Once: Unified, Real-Time Object Detection}. Proceedings of the IEEE Conference on Computer Vision and Pattern Recognition (CVPR).
    \item Banks, A., \& Gupta, R. (2014). \textit{MQTT Version 3.1.1}. OASIS Standard.
    \item Foundation, R. P. (2023). \textit{Raspberry Pi 4 Model B Datasheet}. Raspberry Pi Ltd.
    \item Tiangolo, S. (2019). \textit{FastAPI: High performance, easy to learn, fast to code, ready for production}. https://fastapi.tiangolo.com/
    \item Expo. (2024). \textit{Expo Documentation: Create Universal Native Apps}. https://docs.expo.dev/
    \item Timescale. (2024). \textit{TimescaleDB: PostgreSQL for Time-Series}. https://www.timescale.com/
    \item Mushtaq, Z., \& Su, S. F. (2020). \textit{Environmental sound classification using a regularized deep convolutional neural network with data augmentation}. Applied Acoustics, 167, 107389.
\end{enumerate}

\FloatBarrier

% =================================================================================
%                                   APPENDICES
% =================================================================================
\newpage
\section*{APPENDICES}
\addcontentsline{toc}{section}{Appendices}

\subsection*{Appendix A: Source Code Snippets}

\textbf{1. Audio Producer Thread (Zero Blind Spots)}
\begin{lstlisting}[language=Python]
import threading
import queue
import pyaudio

class AudioProducer(threading.Thread):
    def __init__(self, audio_queue, chunk_size=1024):
        super().__init__()
        self.queue = audio_queue
        self.chunk_size = chunk_size
        self.running = True
        self.pa = pyaudio.PyAudio()

    def run(self):
        stream = self.pa.open(format=pyaudio.paFloat32,
                              channels=1,
                              rate=16000,
                              input=True,
                              frames_per_buffer=self.chunk_size)
        while self.running:
            try:
                data = stream.read(self.chunk_size, exception_on_overflow=False)
                if not self.queue.full():
                    self.queue.put(data)
            except Exception as e:
                print(f"Error recording: {e}")
        stream.stop_stream()
        stream.close()
\end{lstlisting}

\FloatBarrier

\newpage
\textbf{2. FastAPI WebSocket Manager}
\begin{lstlisting}[language=Python]
class ConnectionManager:
    def __init__(self):
        self.active_connections: List[WebSocket] = []

    async def connect(self, websocket: WebSocket):
        await websocket.accept()
        self.active_connections.append(websocket)

    async def broadcast(self, message: str):
        for connection in self.active_connections:
            await connection.send_text(message)
\end{lstlisting}

\FloatBarrier

\subsection*{Appendix B: Project Timeline}
\begin{table}[H]
    \centering
    \begin{tabular}{|l|l|l|}
    \hline
    \textbf{Week} & \textbf{Activity} & \textbf{Outcome} \\
    \hline
    1-2 & Orientation & Hardware selection, Environment setup \\
    \hline
    3-4 & Architecture Design & Multi-threading design, Database Schema \\
    \hline
    5-6 & AI Model & Data augmentation, YOLO training \\
    \hline
    7-8 & Implementation & Backend API, Mobile App UI, MQTT Integration \\
    \hline
    9 & Testing & Stress test, Latency measurement \\
    \hline
    10 & Documentation & Final Report writing \\
    \hline
    \end{tabular}
    \caption{Internship Project Timeline}
\end{table}

\FloatBarrier