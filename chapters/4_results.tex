\newpage
\section{RESULTS AND EVALUATION}

\subsection{System Performance}
The performance of the system was evaluated across all three tiers: Edge, Backend, and Mobile Application.

\subsubsection{Edge Device Efficiency}
The optimization strategies applied to the Raspberry Pi 4 yielded significant improvements over the previous iteration (v1.0).
\begin{table}[H]
    \centering
    \caption{Edge System Performance Metrics}
    \label{tab:performance}
    \begin{tabular}{|l|l|l|}
    \hline
    \textbf{Metric} & \textbf{v1.0 (Baseline)} & \textbf{v2.2 (Optimized)} \\
    \hline
    CPU Usage & 80\% (Spikes) & 45\% - 60\% (Stable) \\
    \hline
    RAM Usage & 600MB & 450MB \\
    \hline
    Inference Latency & 3500ms (Disk I/O bound) & 150ms (In-Memory) \\
    \hline
    End-to-End Latency & > 5 seconds & < 2 seconds \\
    \hline
    \end{tabular}
\end{table}
The "Zero Disk I/O" approach successfully eliminated the bottleneck, allowing for continuous monitoring without "blind spots."

\subsubsection{Backend and Database Performance}
The integration of TimescaleDB demonstrated superior performance for time-series queries compared to standard relational tables.
\begin{itemize}
    \item \textbf{Ingestion Rate:} The system successfully handled concurrent MQTT streams from simulated multiple devices without data loss.
    \item \textbf{Query Speed:} Aggregating 24 hours of temperature data (approx. 86,400 points) for the mobile chart took less than 50ms, ensuring a responsive user experience.
\end{itemize}

\subsection{AI Model Accuracy}
The YOLO model, trained on the "Donate-A-Cry" dataset augmented with domestic noise, achieved the following metrics:
\begin{itemize}
    \item \textbf{Precision (92\%):} High confidence in positive alerts, minimizing unnecessary parental anxiety.
    \item \textbf{Recall (88\%):} The system detects the majority of distress events, though some low-intensity whimpers may be missed.
    \item \textbf{F1-Score (0.90):} Indicates a balanced performance suitable for a consumer safety device.
\end{itemize}

\subsection{Application Stability and User Experience}
A 48-hour continuous stress test was conducted to verify the stability of the mobile application and WebSocket connection.
\begin{itemize}
    \item \textbf{Connection Recovery:} The application successfully re-established WebSocket connections within 3 seconds following a simulated network dropout.
    \item \textbf{Alert Latency:} The average time from a cry event occurring at the Edge to the notification appearing on the Mobile App was measured at 1.8 seconds, well within the acceptable range for real-time monitoring.
    \item \textbf{Battery Impact:} The app's background service (listening for alerts) consumed approximately 5\% battery over a 12-hour period, deemed efficient for a monitoring application.
\end{itemize}
